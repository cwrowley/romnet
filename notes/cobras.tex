\documentclass[11pt]{article}
\usepackage[margin=1in]{geometry}
\usepackage{amsmath}
\usepackage{amssymb}

\begin{document}
\section{Problem formulation}
\label{sec:problem-formulation}

Suppose a state $x(t)\in\mathbb{R}^n$ evolves according to the dynamics
\begin{equation}
  \label{eq:1}
  x(t+1) = f(x(t)),
\end{equation}
with outputs given by
\begin{equation}
  \label{eq:2}
  y(t) = g(x(t)),
\end{equation}
where $y(t)\in\mathbb{R}^m$.

We define an output sequence map $F:x(t)\mapsto
(y(t),y(t+1),\ldots,y(t+{L}))$.  That is,
\begin{equation}
  \label{eq:3}
  F(x) = \big(g(x),g(f(x)),\ldots,g(f^{L}(x))\big)
\end{equation}

\section{GAP loss function}
\label{sec:gap-loss-function}

Suppose we wish to measure how close two states $x$ and $z$ are, in terms of how
close their corresponding output sequences are.  That is, we are interested in
\begin{equation}
  \label{eq:4}
  \|F(x) - F(z)\|.
\end{equation}
In practice, we may be interested in evaluating this quantity for a fixed $x$,
and many different choices of $z$ (for instance, if we are comparing different
approximations of $x$, or training a neural network).  Can we approximate this
quantity in a way that does not involve computing the history $F(z)$ for
every $z$?  If $z$ is close to $x$, we have
\begin{equation}
  \label{eq:5}
  F(z) = F(x) + DF(x)\cdot (z-x) + O(\|z-x\|^2),
\end{equation}
so
\begin{equation}
  \label{eq:6}
  \|F(x) - F(z)\| = \|DF(x)\cdot(z-x)\| + O(\|z-x\|^2)
\end{equation}

In Sam's COBRAs paper, he considered approximating the gradient above by
sampling.  We choose a random vector $\eta\in\mathbb{R}^m$ and a random time
$\tau\in[0,\ldots,L]$, and consider, instead of~(\ref{eq:4}),
\begin{equation}
  \label{eq:7}
  \ell(x) = \big|\eta^T \big(F_\tau(x) - F_\tau(z)\big)\big|,
\end{equation}
where $F_\tau(x)=g(f^\tau(x))$ denotes the $\tau$-th component of $F$.  Then to
first order in $z-x$, and writing $f^\tau(x)=x(\tau)$, we have
\begin{align}
  \label{eq:8}
  \ell(x) &= \big|\eta^T DF_\tau(x) (z-x)\big|\\
          & = \big|\eta^T Dg(x(\tau)) \cdot Df(x(\tau-1)) \cdot Df(x(\tau-2))\cdots
            Df(x(0))\cdot (z-x)\big|
\end{align}
Now, given $x(t)$ for $t=0,\ldots,\tau$, define
\begin{align}
  \lambda(\tau) &= Dg(x(\tau))^T \eta,\\
  \lambda(t) &= Df(x(t))^T \lambda(t+1),\qquad t=0,\ldots,\tau-1.
\end{align}
Then
\begin{equation}
  \label{eq:9}
  \ell(x) = \big|\lambda(0)^T (z - x)\big|.
\end{equation}

\end{document}
