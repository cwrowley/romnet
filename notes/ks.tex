\documentclass[11pt]{article}

\usepackage[margin=1in]{geometry}
\usepackage{amsmath}
\usepackage{amssymb}

\title{Kuramoto-Sivishinsky equation}
\begin{document}
The Kuramoto-Sivashinsky equation is given by

\begin{equation}
  \label{eq:1}
  u_t + u u_x + u_{xx} + u_{xxxx} = 0,\qquad x \in [0,L]
\end{equation}
with periodic boundary conditions.  Let us approximate $u(x,t)$ in terms of the
first $n$ Fourier modes:
\begin{equation}
  \label{eq:2}
  u(x,t) = \sum_{k=-n}^n a_k(t) e^{2\pi ikx/ L}.
\end{equation}
Because $u(x,t)$ is real, we have
\begin{equation}
  \label{eq:3}
  a_{-k} = \overline{a_k},\qquad k=-n,\ldots,n.
\end{equation}
If we let $\hat k = 2\pi k/L$, the linear terms become
\[u_{xx} + u_{xxxx} = \sum_k (-\hat k^2 + \hat k^4) a_k(t) e^{2\pi i k x/L},\]
and the nonlinear terms may be written as the convolution
\begin{align*}
  u u_x
  &= \bigg(\sum_{j=-n}^n a_j(t) e^{2\pi i jx/L}\bigg)\bigg(\sum_{k=-n}^n
    i\hat k a_k(t) e^{2\pi ikx/L}\bigg)\\
  &= \sum_{j,k=-n}^n i\hat k a_j(t) a_k(t) e^{2\pi i(j+k)x/L}.
\end{align*}
\end{document}
